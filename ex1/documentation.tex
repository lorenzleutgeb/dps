\documentclass[a4paper]{article}

\usepackage[utf8]{inputenc}
\usepackage[a4paper, top=2cm, bottom=2cm, left=5mm, right=5mm]{geometry}
\usepackage[hidelinks]{hyperref}
\usepackage{graphicx}
%\usepackage[urw-garamond]{mathdesign}
%\usepackage{ebgaramond}
\usepackage[T1]{fontenc}
\usepackage{csquotes}
\usepackage{natbib}
%\usepackage{amsfonts}

\begin{document}

\begin{center}
	\textbf{\Large{184.701 UE Declarative Problem Solving}}\\[3mm]
	
	\large{Exercise 1: Satisfiability}\\[7mm]

	Lorenz Leutgeb
	
	\texttt{e1127842@student.tuwien.ac.at}
\end{center}

Solutions to the Round Robin Problem \cite{DBLP:conf/aaai/BejarM00} obtained using PicoSAT are exhibited in tables \ref{tab8}, \ref{tab10} and \ref{tab14}.

Figure \ref{chart} shows the execution times (wall time, not CPU time) for PicoSAT, Plingeling and UBCSAT for $n = 4, \ldots, 16$ (resp.). Note that I used a logarithmic scale to account for the explosive combinatorics of the problem \cite{Mcaloon97sportsleague}. Execution times have been averaged over three runs on a machine with an Intel{\textregistered } Core{\texttrademark} i7-4870HQ CPU at 2.50GHz and 16GB main memory with Linux 4.8.13 kernel. Timeout was set to 60 seconds.

UBCSAT does not always terminate. Quoting \cite{DBLP:conf/sat/TompkinsH04}: \enquote{The search process is terminated when a \emph{termination condition} is satisfied; this is typically the case when either a solution, \emph{i.e.}, a satisfying assignment of $F$, has been found or when a given bound on the run-time, which is also referred to as \emph{cutoff time} and which may be measured in search steps or CPU time, has been reached or exceeded.} (Note that here $F$ is the input formula.)

\begin{table}[H]
\begin{center}
\begin{tabular}{|l||c|c|c|c|c|c|c|}
\hline
&Wk.~1&Wk.~2&Wk.~3&Wk.~4&Wk.~5&Wk.~6&Wk.~7\\
\hline\hline
Fld.~1&(6,7)&(5,7)&(4,6)&(3,8)&(3,4)&(1,5)&(2,8)\\\hline
Fld.~2&(1,4)&(2,4)&(2,5)&(1,7)&(7,8)&(3,6)&(3,5)\\\hline
Fld.~3&(2,3)&(1,3)&(1,8)&(2,6)&(5,6)&(4,8)&(4,7)\\\hline
Fld.~4&(5,8)&(6,8)&(3,7)&(4,5)&(1,2)&(2,7)&(1,6)\\\hline
\end{tabular}
\end{center}
\caption{A solution for $n=8$.}
\label{tab8}
\end{table}

\begin{table}[H]
\begin{center}
\begin{tabular}{|l||c|c|c|c|c|c|c|c|c|}
\hline
&Wk.~1&Wk.~2&Wk.~3&Wk.~4&Wk.~5&Wk.~6&Wk.~7&Wk.~8&Wk.~9\\
\hline\hline
Fld.~1&(1,9)&(7,10)&(2,10)&(6,9)&(1,2)&(3,5)&(6,8)&(4,7)&(3,8)\\\hline
Fld.~2&(2,8)&(5,8)&(7,9)&(4,10)&(6,7)&(2,6)&(3,9)&(1,5)&(1,4)\\\hline
Fld.~3&(4,6)&(3,4)&(1,8)&(1,3)&(5,9)&(8,10)&(5,7)&(6,10)&(2,7)\\\hline
Fld.~4&(3,7)&(1,6)&(4,5)&(2,5)&(3,10)&(1,7)&(2,4)&(8,9)&(9,10)\\\hline
Fld.~5&(5,10)&(2,9)&(3,6)&(7,8)&(4,8)&(4,9)&(1,10)&(2,3)&(5,6)\\\hline
\end{tabular}
\end{center}
\caption{A solution for $n=10$.}
\label{tab10}
\end{table}

\begin{table}[H]
\begin{center}
\begin{tabular}{|l||c|c|c|c|c|c|c|c|c|c|c|c|c|}
\hline
&Wk.~1&Wk.~2&Wk.~3&Wk.~4&Wk.~5&Wk.~6&Wk.~7&Wk.~8&Wk.~9&Wk.~10&Wk.~11&Wk.~12&Wk.~13\\
\hline\hline
Fld.~1&(2,7)&(6,9)&(3,8)&(4,12)&(8,11)&(1,3)&(4,6)&(12,13)&(5,14)&(7,11)&(9,14)&(1,10)&(5,13)\\\hline
Fld.~2&(4,11)&(2,10)&(4,5)&(2,6)&(6,12)&(10,13)&(1,14)&(1,9)&(3,7)&(5,9)&(8,12)&(8,13)&(3,11)\\\hline
Fld.~3&(5,12)&(1,5)&(6,11)&(9,10)&(7,10)&(8,14)&(7,12)&(11,14)&(1,13)&(2,3)&(3,6)&(2,4)&(4,8)\\\hline
Fld.~4&(8,9)&(4,13)&(2,9)&(7,14)&(1,2)&(5,6)&(11,13)&(5,10)&(6,8)&(12,14)&(10,11)&(3,12)&(1,7)\\\hline
Fld.~5&(1,6)&(3,14)&(1,12)&(3,13)&(9,13)&(2,12)&(8,10)&(6,7)&(4,9)&(4,10)&(5,7)&(5,11)&(2,14)\\\hline
Fld.~6&(3,10)&(7,8)&(7,13)&(1,11)&(4,14)&(9,11)&(2,5)&(3,4)&(10,12)&(1,8)&(2,13)&(6,14)&(9,12)\\\hline
Fld.~7&(13,14)&(11,12)&(10,14)&(5,8)&(3,5)&(4,7)&(3,9)&(2,8)&(2,11)&(6,13)&(1,4)&(7,9)&(6,10)\\\hline
\end{tabular}
\end{center}
\caption{A solution for $n=14$.}
\label{tab14}
\end{table}

\begin{figure}[H]
\begin{center}
\includegraphics[width=1.0\textwidth]{chart}
\end{center}
\caption{Chart showing the execution time of three different solvers for growing instance sizes.}
\label{chart}
\end{figure}

\begin{table}[H]
\begin{center}
\begin{tabular}{|l|ccccccc|l|}
\hline
          &  4& 6& 8& 10&   12& 14&16&Version\\\hline
\hline
\texttt{picosat}   &n/a& 4&22& 833&22446&t/o&t/o&965\\
\texttt{plingeling}&n/a&16&88& 106&16327&t/o&t/o&ayv\\
\texttt{ubcsat -cutoff max -timeout 60 -alg saps -solve}   &n/a& 5&19&1124&26115&t/o&t/o&1.1.0\\
\hline
\end{tabular}
\end{center}
\caption{Execution times of three different solvers for growing instance sizes.}
\label{charttab}
\end{table}

\bibliographystyle{alpha}
\bibliography{references}

\end{document}